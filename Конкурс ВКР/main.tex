\documentclass{article}

\usepackage[T1,T2A]{fontenc}
\usepackage[utf8]{inputenc}
\usepackage[english,russian]{babel}

\usepackage{pdfpages}
\usepackage{multirow}

\usepackage{caption}

\usepackage{amsmath}
\usepackage[hidelinks]{hyperref}


\usepackage{graphicx}%Вставка картинок
\graphicspath{{noiseimages/}}

\usepackage{float}%'Плавающие' картинки
\usepackage{wrapfig}%Обтекание фигур (таблиц, картинок и прочего)

% Параграфы перед главами, но их не будет перед сабсекциями
\renewcommand{\thesection}{\S\arabic{section}}
\renewcommand{\thesubsection}{\arabic{section}.\arabic{subsection}}

\makeatletter
\def\@biblabel#1{#1. }
\makeatother

\setlength{\emergencystretch}{10pt}

\begin{document}

\section{МАТЕМАТИЧСКОЕ МОДЕЛИРОВАНИЕ УПРАВЛЯЮЩИХ ПАРАМЕТРОВ ЭЛЕКТРОЛИЗНОЙ ВАННЫ}

Алюминиевая промышленность занимает в экономике России одно из ведущих мест. Эффективное управление промышленным процессом электролиза алюминия требует внедрения автоматизированных систем управления – АСУТП, в которых в качестве начальных данных используются управляющие параметры, полученные в результате анализа физико-химических процессов, протекающих в электролизной ванне.  Высокие температуры и агрессивность среды, в которых происходит процесс электролиза алюминия, не позволяют провести измерения большинства параметров управления. Наиболее перспективным направлением является разработка новых алгоритмов управления, построенных на понимании и моделировании технологического процесса электролитического получения алюминия. Это дает возможность автоматизировать отдельные контуры управления и оказывать поддержку технологу при принятии решений с целью увеличения выхода алюминия по току. 

Для определения двух важнейших управляющих параметров (выхода по току и потерь выхода по току), необходимы замеры межполюсного расстояния, плотности анодного тока и формы поверхности раздела сред металл-электролит, которые на практике определяются весьма приблизительно. В свою очередь трёхмерная трёхфазная математическая модель процесса электролиза алюминия, на основе которой был реализован вычислительный комплекс, позволяющий достаточно точно вычислить величины, необходимые для определения управляющих параметров, учитывает взаимосвязь всех основных динамических процессов, происходящих в электролизной ванне.

Для вычисления выхода по току необходимы теоретическое количество алюминия, которое должно получиться при электролизе и вычисляемое по закону Фарадея и реально
произведенное количество произведенного металла.
Однако количество произведённого металла  измеряется толко раз в сутки, тогда как состояния МГД нестабильности могут развиваться за доли секунды.

Экспериментальные исследования показывают, что выход по току $\eta$ зависит от большого числа параметров: температуры, межполюсного расстояния (МПР) - расстояния между подошвой анода и границей раздела сред металл-электролит, плотности тока, геометрии рабочего пространства электролизёра, электромагнитных сил [\ref{litlink:belo}].

На практике в АСУТП применяется следующая эмпирическая формула разработанная институтом ВАМИ (Всероссийский Алюминиево-магниевый Институт).
При этом значения анодной плотности тока, температуры электролита и МПР определяются экспериментально в нескольких отдельных точках рабочего пространства ванны.
Таким образом точность вычисления выхода по току по эмперической формуле напрямую зависит от точности входящих в неё параметров. Поскольку на практике значениям этих параметров соответствует большая погрешность, то величина $\eta$ по данной формуле вычисляется с большой погрешностью. Трёхмерное трёхфазное математическое моделирование [\ref{litlink:kalmykov}] позволяет исследовать распределение в рабочем пространстве ванны всех необходимых параметров с достаточно большой точностью и изучить распределение $\eta$ и $\Delta\eta$, на основе которого можно сделать достоверную оценку эффективности производства алюминия в различные моменты времени.

В зависимости от режима работы электролизной ванны процесс электролиза может быть МГД-стабилен или МГД-нестабилен. В случае МГД-нестабильной работы ванны расстояние между анодом и катодом может быть критически мало, что ведёт к резкому уменьшению выхода по току. В работе приводятся результаты вычисления величины выхода по току в разные моменты времени при МГД-стабильности и МГД-нестабильности ванны. 

Из проведенных в работе вычислений видно, что выход по току имеет общую тенденцию на уменьшение при искривлении поверхности. 

В работе [\ref{litlink:derkach2}] представлена полуэмпирическая формула выхода по току, опирающаяся на распределения МПР по горизонтальному срезу ванны.

Однако практическая реализация полуэмпирической формулы зависит от точности определения поверхности раздела сред металл-электролит, а также требует достаточно точного вычисления поверхностного интеграла. 

При этом величина МПР, как и в эмперической формуле, определяется очень грубо, поэтому практическое ее использование весьма затруднительно. Более того, как показали проведённые в работе исследования, полуэмперическая формула является противоречивой. Поэтому, ее применение не является целесообразным.

В работе предлагается модифицированная формула потерь выхода по току 

\begin{equation} \label{eq:modf2}
\Delta \eta = (1- \eta_0) \cdot \frac{1}{S} \cdot \int\limits_Z \frac{l(x,y) ds}{H(x,y)},
\end{equation}
в которой значения величин $l(x,y)$,$ H(x,y)$ в каждый момент времени определяются при помощи трёхмерного математического моделирования.

Для вычисления модифицированной формулы, в работе был разработан метод простой триангуляции. Его точность была ииследована на практике методом
сгущающихся сеток и имеет второй порядок.

Численный расчёт значений потерь выхода по току в случаях МГД-стабильной работы ванны, при выемке анодов и анодном эффекте показал, что наибольшие потери выхода по току соответствуют анодному эффекту, а наименьшие – МГД-стабильному режиму работы ванны. Этот вывод соответствует практическим наблюдениям.

Было показана закономерность увеличения потерь по току при МГД-нестабильности. Причем потери при выемке анодов меньше, чем при анодном эффекте. В то же время при стабилизации поверхности раздела сред потери уменьшаются для любого режима работы ванны.

Также с помощью модифицированной формулы \ref{eq:modeq2} были получены распределения потерь по току в электролизной ванне. Анализ проведённых численных экспериментов позволяет сделать выводы о корреляции результатов расчетов и результатов лабораторных исследований стабильности режима работы ванны.

Эффективность работы электролизной ванны диктует необходимость развития инструментов управления технологическим процессом с целью быстрого реагирования на изменения режима работы ванны. Значения основных управляющих параметров (криолитовое отношение, выход по току, потери по току) наглядно отражают физико-химические процессы, происходящие в среде электролита, и позволяют судить о применении тех или иных технологических решений с целью стабилизации режимов работы ванны. Предложенные в работе методы вычисления управляющих параметров ванны позволяют вычислять и своевременно реагировать на их изменения, что в свою очередь способно значительно увеличить эффективность и прибыльность производства.

\end{document}