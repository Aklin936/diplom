\begin{vkrthesis}%
{
Вычисление управляющих параметров работы электролизной ванны
}%
{Ненахов~Н.\,Д.}%

\VKRAuthorDetailsSupervisorConsultant%
{Ненеахов Нил Денисович}%
{Кафедра вычислительных методов}%
{s02200678@cmc.msu.ru, nenakhov.neil@mail.ru}%
{Савенкова Надежда Петровна}%
{д.ф.-м.н.}%
{вед. науч. сотр.}%
{Бычков Владимир Львович}%
{д.ф.-м.н.}%
{вед. науч. сотр.}%
{}%

% Здесь можно для удобства переопределить некоторые команды
\newcommand{\tbs}{\textbackslash}

Эффективное управление промышленным процессом электролиза алюминия требует внедрения автоматизированных систем управления – АСУТП.  Высокие температуры и агрессивность среды, в которой происходит процесс электролиза алюминия, не позволяют провести измерения большинства параметров управления. Наиболее перспективным направлением является разработка новых алгоритмов управления, построенных на понимании и моделировании технологического процесса электролитического получения алюминия. 

Для определения выхода по току и потерь выхода по току ($\eta, \Delta \eta$ соответственно), которые являются важными управляющими параметрами, необходимы замеры межполюсного расстояния, плотности анодного тока и формы поверхности раздела сред металл -- электролит, которые на практике определяются весьма приблизительно \ref{litlink:belo}. В свою очередь, трёхмерная трёхфазная математическая модель процесса электролиза алюминия \ref{litlink:kalmykov}, на основе которой был реализован вычислительный комплекс, позволяет достаточно точно вычислить величины, необходимые для определения управляющих параметров, и учитывает взаимосвязь всех основных динамических процессов, происходящих в электролизной ванне.

В зависимости от режима работы электролизной ванны процесс электролиза может быть МГД -- стабилен или МГД -- нестабилен. В случае МГД -- нестабильной работы ванны расстояние между анодом и катодом может быть критически мало, что ведёт к резкому уменьшению выхода по току. В работе приводятся результаты вычисления величины выхода по току в разные моменты времени при МГД -- стабильности и МГД -- нестабильности ванны. 

На практике в АСУТП применяется эмпирическая формула, разработанная институтом ВАМИ \ref{litlink:VAMI} (Всероссийский алюминиево --- магниевый институт).
При этом значения параметров определяются экспериментально в нескольких отдельных точках рабочего пространства ванны.
Поскольку на практике значениям параметров, входящих в формулу, соответствует большая погрешность, то величина выхода по току $\eta$ по данной формуле вычисляется с значительной погрешностью.

Из проведенных в работе вычислений следует, что выход по току имеет общую тенденцию на уменьшение при искривлении поверхности. 

В работе \ref{litlink:derkach2} представлена полуэмпирическая формула выхода по току, опирающаяся на распределения МПР по горизонтальному срезу ванны.
Однако ее практическая реализация зависит от точности определения поверхности раздела сред металл -- электролит, а также требует достаточно точного вычисления поверхностного интеграла. 

При этом величина МПР, как и в эмпирической формуле, определяется очень грубо, поэтому практическое ее использование весьма затруднительно. Более того, как показали проведенные в работе исследования, полуэмпирическая формула является противоречивой. Поэтому ее применение не является целесообразным.
В работе предлагается модифицированная формула потерь выхода по току $\Delta \eta$:

\vspace{-2em}
\begin{equation} \label{eq:modf2}
\Delta \eta = (1- \eta_0) \cdot \frac{1}{S} \cdot \int\limits_Z \frac{l(x,y) ds}{H(x,y)},
\end{equation}
в которой значения величин межполюсного расстояния $l(x,y)$ и глубины жидкого металла $ H(x,y)$ в каждый момент времени определяются при помощи трёхмерного математического моделирования.

Для вычисления по модифицированной формуле (\ref{eq:modf2}) в работе предложен метод простой триангуляции второго порядка точности. Точность исследована на сгущающихся сетках.
Численный расчёт в плоскости электролизной ванны значений потерь выхода по току в случаях МГД -- стабильной работы ванны, при выемке анодов и анодном эффекте показал, что наибольшие потери выхода по току соответствуют анодному эффекту, а наименьшие –- МГД -- стабильному режиму работы ванны. Этот вывод соответствует практическим наблюдениям.

Также с помощью модифицированной формулы (\ref{eq:modf2}) получены распределения потерь по току в электролизной ванне. Анализ проведённых численных экспериментов позволяет сделать выводы о качественном соответствии результатов расчетов и результатов лабораторных исследований стабильности режима работы ванны.

\begin{vkrreferences}

\item
\label{litlink:belo}
Белолипецкий~В.\,М., Пискажова~Т.\,В.
Математическое моделирование процесса электролитического получения алюминия. Решение задач управления технологией: монография~//
(Красноярск, 2012\,г.). --- Сиб. федер. ун-т. ---
С.\,13--58.

\item
\label{litlink:kalmykov}
Калмыков~А.\,В.
Математическое моделирование влияния процессов тепломассопереноса на МГД-стабильность алюминиевого электролизёра~// Москва: Московский государственный университет имени М.В. Ломоносова. Факультет вычислительной математики и кибернетики. Кафедра вычислительных методов. Диссертация. 2017 г.

\item
\label{litlink:VAMI}
Громова~Б.\,С.
Тепловые процессы в электролизерах и миксерах алюминиевого производства.~// М.: 1998. С.\,322.

\item
\label{litlink:derkach2}
Деркач~А.\,С.
Влияние нестабильности тока серии на технологический режим алюминиевых электролизеров~// Цветные металлы. – № 3. – 1967. 39–-40~с.
\end{vkrreferences}
\end{vkrthesis}
