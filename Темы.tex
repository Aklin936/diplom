\documentclass{article}

\usepackage[T2A]{fontenc}
\usepackage[utf8]{inputenc}
\usepackage[russian]{babel}

\usepackage{pdfpages}
\usepackage{multirow}

\usepackage{caption}

\usepackage{amsmath}
\usepackage[hidelinks]{hyperref}


\usepackage{graphicx}%Вставка картинок
\graphicspath{{noiseimages/}}

\usepackage{float}%"Плавающие" картинки
\usepackage{wrapfig}%Обтекание фигур (таблиц, картинок и прочего)

\makeatletter
\def\@biblabel#1{#1. }
\makeatother

\setlength{\emergencystretch}{10pt}

\begin{document}

Практика:
Исследование метода триангуляции приближенного вычисления значения поверхностного интеграла.

Обзор книг по триангуляции.
Это фрагмент решения большей задачи. Привязаны к сетке вычислительного комплекса, поэтому предлагается данный метод.
Исследование точности метода (с аналитическим вычислением и на сгущающейся сетке).

Диплом:
Вычисление управляющих параметров работы электролизной ванны.
Calculation of electrolysis bath control parameters

\end{document}