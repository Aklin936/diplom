\subsection{Анализ полученных значений}

В разделе \ref{sec:teor} приведены результаты расчётов $\Delta \eta$ для конкретных поверхностей с помощью квадратурной формулы трапеций. Эти значения для поверхностей заданных формулами \ref{eq:surf1}, \ref{eq:surf2} приведены в таблицах \ref{table:kvadr1}, \ref{table:kvadr2} соответственно, для $h=100$.

\begin{table}[ht]
	\begin{tabular}{|c|c c c|}
	\hline
	МПР \textbackslash Метод	& Аналит.		& Квадратурная с шагом h	& Квадратурная с шагом h/2\\
	\hline
	1 см						& 0,002689559	& 0,00269251071671692		& 0,00269101678196707		\\
	6 см						& 0,016137327	& 0,0161550643003017		& 0,0161461006918025		\\
	11 см						& 0,02958512	& 0,0296176178838864		& 0,029601184601638			\\
	\hline
	Переменный МПР				& -				& 0,00962397334561231		& 0,00954927660811893		\\
	\hline
	\end{tabular}
	\caption{Значения полученные в разделе \ref{ex1m}\label{table:kvadr1}}
\end{table}

\begin{table}[ht]
	\begin{tabular}{|c|c c|}
	\hline
	МПР \textbackslash Метод	& Квадратурная с шагом h	& Квадратурная с шагом h/2	\\
	\hline
	1 см						& 0,00287021386034949		& 0,00287051817440043		\\
	6 см						& 0,0172212831620969		& 0,0172231090464026		\\
	11 см						& 0,0315723524638443		& 0,0315756999184048		\\
	\hline
	Переменный МПР				& 0,0182523533315609		& 0,0182536360564266		\\
	\hline
	\end{tabular}
	\caption{Значения полученные в разделе \ref{ex2m} \label{table:kvadr2}}
\end{table}

Воспользовавшись формулой \ref{eq:runge3} оценим порядок точности данной квадратурной формулы, результат приведён в таблице \ref{table:porTochKvadr}.

\begin{table}[ht]
	\begin{tabular}{|c c|}
	\hline
	МПР		& Порядок точности \\
	\hline
	1 см	& 1,01777930991882	\\
	6 см	& 1,01553050740999	\\
	11 см	& 1,01646057347639	\\
	\hline
	\end{tabular}
	\caption{Порядок точности формулы трапеций. \label{table:porTochKvadr}}
\end{table}

\newpage

\end{comment}